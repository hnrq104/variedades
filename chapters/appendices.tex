\renewcommand{\thesection}{\Alph{section}}
\chapter{Appendices}
\section{Point-Set Topology}
\subsection{Exercises}

\begin{customprob}{A.5}
\end{customprob}
\begin{proof}
	We know, that for every $\alpha$, $I_\alpha \subset \sum_\beta I_\beta$, so we find that
	$$Z\bigg(\sum_\alpha I_\alpha \bigg) \subset Z(I_\alpha)$$
	As this is valid for every $\alpha$,
	$$ Z\bigg(\sum_\alpha I_\alpha \bigg) \subset \bigcap_\alpha Z(I_\alpha)$$
	The other side is true, for $p$ is a zero of every element of the basis, then it is a common zero for all elements in the ideal.

	For the second part, notice, by looking at generators that $Z(I) \cup Z(J) \subset Z(IJ)$.
	Now consider $p \in Z(IJ)$, then we know that $p$ is a common zero of all pairs $f_ig_j$, with
	$f_i \in I$ and $g_j \in J$. Now suppose that there are $f \in I, g \in J$ s.t $f(p) \neq 0$ and $g(p) \neq 0$.
	As we are in a domain, $f(p)g(p) \neq 0$, as such $p \not \in Z(IJ)$.
\end{proof}

\begin{customprob}{A.33}
\end{customprob}
\begin{proof}
	Let $F_1, F_2$ be two closed sets, by proposition A.32, they are compact. For each element $p \in F_2$, apply proposition A.31
	finding open sets $U_p,V_p$ s.t $F_1 \subset U_p$ and $p \in V_p$. Now, as $F_2$ is compact, take a finite cover
	$$F_2 \subset \bigcup_\alpha V_\alpha = B$$
	and a finite intersection:
	$$F_1 \subset \bigcap_\alpha U_\alpha = A$$
	Notice $A\cap B = \varnothing$, as:
	$$A \cap B = \bigcup_\alpha (V_\alpha \cap A) \subset \bigcup_\alpha (V_\alpha \cap U_\alpha) = \varnothing$$
\end{proof}

\begin{customprob}{A.37}
\end{customprob}
\begin{proof}
	Given a covering of the union, take a finite subcover for each of the finite compacts. There are only finitely many open sets chosen.
\end{proof}

\begin{customprob}{A.53}
\end{customprob}
\begin{enumerate}[label=(\alph*)]
	\item \begin{proof}
		      $\overline{A} \cup \overline{B}$ is a closed set that contains $A \cup B$, so $\overline{A \cup B} \subseteq \overline{A} \cup \overline{B}$.
		      But also, as $A \subset A \cup B$ and $B \subset A \cup B$, we have $\overline{A} \subset \overline{A \cup B}$ and
		      $\overline{B} \subset \overline{A \cup B}$. So  $\overline{A} \cup \overline{B} \subset \overline{A \cup B}$.
	      \end{proof}
	\item \begin{proof}
		      Just as before, $\overline{A} \cap \overline{B}$ is a closed set that contains $A \cap B$.
	      \end{proof}
\end{enumerate}

\subsection{Problems}
\begin{problem}
\end{problem}
\begin{proof}
	This is almost tautological. If $(u,v) \in (U_1 \cap U_2) \times (V_1 \cap V_2)$, then $u \in U_1$ with $v \in V_1$ and
	$u \in U_2$ and $v \in V_2$, so $(u,v) \in U_1 \times V_1 \cap U_2 \times V_2$. Similarly, if $(u,v) \in U_1 \times V_1 \cap U_2 \times V_2$,
	then $u \in U_1 \cap U_2$ and $v \in V_1 \cap V_2$, so $(u,v) \in (U_1 \cap U_2) \times (V_1 \cap V_2)$.
\end{proof}

\begin{problem}
\end{problem}

\begin{proof}
	$$(U_1 \cap U_2) \cap (V_1 \cup V_2) = (U_1 \cap U_2 \cap V_1) \cup (U_1 \cap U_2 \cap V_2) = \varnothing$$
\end{proof}

\begin{problem}
\end{problem}

\begin{enumerate}[label=(\alph*)]
	\item \begin{proof}
		      Each $F_i$ is of the form $(U_i)^c$. As such, we find
		      $$\bigcup_{i=1}^{n} F_i = \bigcup_{i=1}^{n} (U_i)^c = \bigg( \bigcap_{i=1}^{n} U_i \bigg)^c$$
		      But, $\bigcap_{i=1}^{n} U_i$ is open, so our finite union is closed.
	      \end{proof}
	\item \begin{proof}
		      By exactly the same expression:
		      $$\bigcap_{i=1}^{n} F_i = \bigcap_{i=1}^{n} (U_i)^c = \bigg( \bigcup_{i=1}^{n} U_i \bigg)^c$$
	      \end{proof}
\end{enumerate}

%only now did I create a command for begin problem end problem, i will use it from now on, i might come back and
% substitute the previous ones later
\prob
\begin{proof}
	For $p \in ]a,a[^n$, $ |p| < \sqrt{n}a$. And if $p \in B(0, \sqrt{n}a)$, then for each coordinate, $|p^i| < \sqrt{n}a$, as such
	$p \in ]\sqrt{n}a,\sqrt{n}a[^n$. Now, to prove the cubes form a basis, we use the definition, for every open set $U$ and $p \in U$,
	choose the ball centered in $p$ contained in $U$ and the open cube centered in $p$ contained in the ball.
\end{proof}

\prob
\begin{proof}
	$A^c$ is open, $B^c$ is open, so $A^c\times Y$ and $X \times B^c$ are open. So $A \times Y$ and $X \times B$ are closed,
	and we have $A \times Y \cap  X \times B = A \times B$ closed.
\end{proof}

\prob
\begin{proof}
	If $S$ is Haussdorf, consider $(a,b) \in S\times S - \Delta$. There are disjoint open sets $A,B$ with $a \in A$ and $b \in B$,
	then $A \times B \subset S\times S - \Delta$ is a open neighborhood of $(a,b)$, as $(a,b)$ were arbitrary, $\Delta$ is closed.
	Now, if $\Delta$ is closed, $S\times S - \Delta$ is open, and if $a \neq b \in S$, $(a,b) \in S\times S - \Delta$ is contained in
	some open neighborhood $A \times B$ (because these sets form a basis for $S\times S$). Take, as  before $A$ and $B$.
\end{proof}

\prob

\begin{proof}
	Consider an open set $A = \bigcup_\alpha U_\alpha \times V_\alpha$. We find that $\pi(A) = \bigcup_\alpha U_\alpha$ which is open.
\end{proof}

\prob
\begin{proof}
	If $f$ is continuous, then the preimage of every open set is open. In particular, given $p$, the preimage of the open ball $B(f(p),\eps)$
	is an open set of $A$, which contains $p$. As $p \in f^{-1}(B(f(p), \eps)$ open, there is $\delta > 0$ s.t. $B(p,\delta) \subset f^{-1}(B(f(p), \eps)$
	as we wanted to show.

	For the converse, sps $f$ satisfies the $\eps$-$\delta$ criterion. Consider $V$ open set of $R^n$, if $V = \varnothing$, then
	we have nothing to prove, as the preimages of $V$ is the empty set, which is open in A. Now let $f(p) \in V$, because $V$ is open, there is $r>0$ s.t.
	$B(f(p), r) \subset V$. By the criterion, there is $\delta > 0$ s.t. $f(B(p,\delta)) \subset B(f(p), r)$, that is, $B(p,\delta) \subset f^{-1}(V)$, which means
	$f^{-1}(V)$ is open - as p was arbitrary.
\end{proof}

\prob
\begin{proof}
	For any function $f:X \to Y$ and set $A \subset Y$, $f^{-1}(A^c) = (f^{-1}(A))^c$. As such, if $f$ is continuous
	and $F = U^c$ is a closed set. $f^{-1}(F) = f^{-1}(U^c) = (f^{-1}(U))^c$ which is closed.
	Absolutely the same proof is valid for proving the assertion from the closed to the open sets.
\end{proof}

\prob
\begin{proof}
	We know that the projections are continuous, as such, if $f$ is continuous, so are their compositions.
	If both projections are continuous, then by noticicng that $f^{-1}(U \times V) = f_1^{-1}(U) \cap f_2^{-1}(V)$ which
	is a finite intersection of open sets, we check that $f$ is continuous. (To generalize to any open set in the product topology, separate it
	into basis elements)
\end{proof}

\prob
\begin{proof}
	Sps $f \times g$ is continuous, then given an open set $U \subset X'$, consider $(f\times g)^{-1}(U \times Y') = f^{-1}(U) \times Y$ which is open
	by the continuity of $f \times g$. By problem A.7, $f^{-1}(U)$ is open. The same argument is valid for $g$.

	If $f,g$ are continuous, and $\bigcup_\alpha U'_\alpha \times V'_\alpha = A$ is an open set of $X'\times Y'$, then $(f \times g)^{-1}(A)
		= f^{-1}(\bigcup_\alpha U'_\alpha) \times g^{-1}(\bigcup_\alpha V'_\alpha)$ is open as well.
\end{proof}

\prob
\begin{proof}
	The inverse function $f^{-1}$ has the closed pre-images property (because $f$ is closed). So by a previous problem $f$ is continuous.
\end{proof}

\prob
\begin{proof}
	Given any covering of the topological space, consider the countable set of basis elements contained in the cover, for
	each such set, choose one set of the covering which contains it.
\end{proof}

\prob
\begin{proof}
	Take a covering of such union, take the union of the finite coverings for each compact set.
\end{proof}

\prob
\begin{proof}
	Notice that this is precisely the definion of connectedness for the subspace when noticing $A = (A\cap U) \cup (A \cap V)$.
\end{proof}

\prob
\begin{proof}
	If $C$ is a connected component and $p \in C$, by the local connected property, there is a connected neighborhood $V_p$, as $C \cup V_p$ is connected
	$C \cup V_p = C$, as such, $V_p \subset C$. So $C$ is open, as $p$ was arbitrary.
\end{proof}

\prob
\begin{proof}
	Clearly, if $U \cap A \neq \varnothing$ then $U \cap \overline{A} \neq \varnothing$.
	Now suppose there is $p \in U \cap \overline{A}$, then as $U$ is open, there is an open neighborhood $V_p \subset U$. But,
	as $p \in \overline{A}$, every such neighborhood intersects $A$, so $V_p \cap A \neq \varnothing$.
\end{proof}

\prob
\begin{proof}
	Clearly a countable basis for the whole set will contain a countable basis for every neighborhood.
\end{proof}

\prob
\begin{proof}
	Sps, for the sake of contradiction that there were two limits, $x$ and $y$. As the space is $Hausdorff$,
	there is a separation $U \cap V = \varnothing$ with $x \in U, y \not \in U$ and $x \not \in V, y \in V$.
	By the property of the limit, eventually the sequence is forever in $U$, never in $V$, contradicting $y$ being a limit point.
\end{proof}

\prob
\begin{proof}
	$cl_S(A) \times Y$ is a closed set of $S \times Y$ that contains $A \times Y$, as such,
	$cl_{S \times Y}(A \times Y) \subset cl_S(A) \times Y$.

	Sps $(a,y) \in cl_S(A) \times Y$, then if $a \in A$, there's nothing to show. Then, suppose $a \in ac(A)$.
	we must show $(a,y) \in ac(A \times Y)$. Take any open neighborhood of $(a,y)$ from the basis of the product topology,
	then $(a,y) \in U\times V$, as such, $U \cap A \neq \varnothing$, and of course $U\times V \cap A \times Y \neq \varnothing$.
\end{proof}

\prob
\begin{enumerate}[label=(\alph*)]
	\item \begin{proof}
		      Follows directly from proposition A.48 and $S = cl(A)$
	      \end{proof}
	\item \begin{proof}
		      From A.18, we saw that any two non-empty open sets intersect. So every neighborhood intersects with A.
	      \end{proof}
\end{enumerate}
