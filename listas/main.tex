\documentclass{article}

\usepackage{amsmath,amssymb,amsthm,bbm,mathtools,calc,verbatim,enumitem,tikz,url,mathrsfs,cite,fullpage,hyperref,bm,marvosym}
\usepackage{dsfont}
\usepackage{float}
\usepackage{subcaption}
%\usepackage{setspace}
\renewcommand{\baselinestretch}{1.1}
\addtolength{\footskip}{\baselineskip/2}

%\usepackage{showlabels}
\usepackage{comment}
\usepackage[english]{babel}

%No caso do livro o Tu, para decoplar seções de capitulos usamos:
% \usepackage{chngcntr}
% \counterwithout{section}{chapter}

\theoremstyle{definition}
\newtheorem{theorem}{Theorem}[section]
\newtheorem{lemma}[theorem]{Lemma}
\newtheorem{corollary}[theorem]{Corollary}
\newtheorem{prop}[theorem]{Proposition}
\newtheorem{observation}[theorem]{Observation}
\newtheorem{construction}[theorem]{Construction}

\newtheorem{definition}[theorem]{Definition}

\newtheorem{conjecture}[theorem]{Conjecture}
\newtheorem{question}[theorem]{Question}
\newtheorem{obs}[theorem]{Observation}
\newtheorem{claim}[theorem]{Claim}
\newtheorem{fact}[theorem]{Fact}
\newtheorem{problem}{Problem}[section]
\newtheorem{exercise}{Exercise}[section]
\newtheorem{remark}[theorem]{Remark}

% my custom problems
\newtheorem{innercustomexercise}{Exercise}
\newenvironment{customexercise}[1]
  {\renewcommand\theinnercustomexercise{#1}\innercustomexercise}
  {\endinnercustomexercise}

\newenvironment{clmproof}[1]{\begin{proof}[Proof of Claim~\ref{#1}]\let\qednow\qedsymbol\renewcommand{\qedsymbol}{}}{\; \qednow \end{proof}}

\newcommand\N{\mathbb{N}}
\newcommand\R{\mathbb{R}}
\newcommand\Z{\mathbb{Z}}
\newcommand\cA{\mathcal{A}}
\newcommand\cB{\mathcal{B}}
\newcommand\cN{\mathcal{N}}
\newcommand\cP{\mathcal{P}}
\newcommand\cQ{\mathcal{Q}}
\newcommand\cZ{\mathcal{Z}}
\newcommand\rN{\tilde{N}}
\newcommand\cT{\mathcal{T}}
\newcommand\cE{\mathcal{E}}
\def\Pr{\mathbb{P}}
\def\cS{\mathcal{S}}
\newcommand\Ex{\mathbb{E}}
\newcommand\id{\hbox{$1\mkern-6.5mu1$}}
\newcommand\lcm{\operatorname{lcm}}
\newcommand\eps{\varepsilon}
\newcommand{\floor}[1]{\lfloor #1 \rfloor}
\newcommand{\ceil}[1]{\lceil #1 \rceil}
\newcommand{\prob}{\begin{problem} \end{problem}}
\newcommand{\exer}{\begin{exercise} \end{exercise}}
\newcommand{\cexer}[1]{\begin{customprob}{#1}\end{customprob}}


\renewcommand{\leq}{\leqslant}
\renewcommand{\geq}{\geqslant}
\renewcommand{\le}{\leqslant}
\renewcommand{\ge}{\geqslant}
\renewcommand{\to}{\rightarrow}
\renewcommand{\Re}{\re}

\def\ds{\displaystyle}

\def\eps{\varepsilon}
\def\p{\partial}

\def\HH{\mathcal{H}}
\def\E{\mathbb{E}}
\def\C{\mathbb{C}}
\def\cM{\mathcal{M}}
\def\cF{\mathcal{F}}
\def\cI{\mathcal{I}}
\def\R{\mathbb{R}}
\def\bS{\mathbb{S}}
\def\bH{\mathbb{H}}
\def\Z{\mathbb{Z}}
\def\N{\mathbb{N}}
\def\PP{\mathbb{P}}
\def\1{\mathbbm{1}}
\def\l{}
\def\k{\kappa}
\def\w{\omega}
\def\s{\sigma}
\def\t{\theta}
\def\a{\alpha}
\def\g{\gamma}
\def\z{\zeta}
\def\zbar{\bar{z}}
\def\<{\langle}
\def\>{\rangle}
%\def\endproof{{\hfill $\square$} }
\def\Xt{\widetilde{X}}
\def\Pt{\widetilde{P}}

\def\cN{\mathcal{N}}
\def\cC{\mathcal{C}}
\def\cD{\mathcal{D}}
\def\cR{\mathcal{R}}
\def\cB{\mathcal{B}}
\def\cG{\mathcal{G}}
\def\EE{\mathbb{E}}
\def\FF{\mathbb{F}}
\def\T{\mathbb{T}}
\def\cA{\mathcal{A}}
\def\cQ{\mathcal{Q}}
\def\cC{\mathcal{C}}
\def\F{\mathbb{F}}
\def\tm{\tilde{\mu}}
\def\ts{\tilde{\sigma}}
\def\Q{\mathcal{Q}}
\def\vp{\varphi}

\DeclareMathOperator\supp{supp}

\hypersetup{
	colorlinks=true,
	linkcolor=blue,
	urlcolor=blue,
}

\pagestyle{plain}
\author{henrique}
\title{Listas de Variedades}

\begin{document}
\maketitle

\tableofcontents
\setcounter{section}{-1}

\section{Introdução e Notação}
Ao decorrer do curso, vou escrever minhas resoluções dos exercícios nesse arquivo. Tem alguns motivos para isso:
\begin{enumerate}
	\item Posso reutilizar resultados passados.
	\item Está tudo organizado se um futuro henrique quiser rever.
	\item Há uma certo senso de completude no final do curso.
\end{enumerate}
Por isso, peço desculpa ao monitor e ao professor se não gostarem desse formato, me avisem que eu posso separar os arquivos.
O código fonte pode ser encontrado em \url{https://github.com/hnrq104/variedades/tree/main/listas}.
\section{Lista 1 (18/08/2025)}

Problemas feitos:
\begin{enumerate}
    \item Exercício \ref{prob:l1:1} : \checkmark
    \item Exercício \ref{prob:l1:2} : \checkmark
    \item Exercício \ref{prob:l1:3} : \checkmark
    \item Exercício \ref{prob:l1:4} : \Frowny
\end{enumerate}

\begin{problem}
\label{prob:l1:1}   
\end{problem}

\begin{proof}
    Defina $(S^1, \mathcal{F})$ a parametrização do círculo pelas projeções esfereográficas. Isto é,
    $$\mathcal{F} = \langle (S^1 - \{(0,1)\}, \pi_N), (S^1 - \{(0,-1)\}, \pi_S)\rangle$$
    Onde $\pi_N : S^1 - \{(0,1)\} \to \R$ e $\pi_S : S^1 - \{(0,-1)\} \to \R$ são as projeções do polo norte e sul respectivamente. Vimos em aula que, com
    essas coordenadas, $(S^1, \mathcal{F})$ é uma variedade $C^\infty$. Defina $\mathcal{G}$ elevando $\mathcal{F}$ ao cubo,
    $$\mathcal{G} = \langle(S^1 - \{(0,1)\}, (\pi_N)^3), (S^1 - \{(0,-1)\}, (\pi_S)^3)\rangle$$
    Afirmo que $\mathcal{G}$ é uma estrutura diferenciável de $S^1$. Isso segue do fato que $\pi_N^3$ e $\pi_S^3$ continuam sendo homeomorfismos 
    e a composição de cartas dão a mesma coisa que em $\mathcal{F}$. Para verificar isso, escreva $s(t) = t^3$, então, no intervalo de definição $\R^*$,
    \begin{align*}
        [(\pi_N)^{3}] \circ [(\pi_S)^{3}]^{-1} (t) &= (s\circ \pi_N) \circ (s \circ \pi_S)^{-1} (t)\\
                                                   &= s \circ \pi_N \circ \pi_S^{-1} \circ s^{-1} (t) \\
                                                   &= s \circ \pi_N \circ \pi_S^{-1} (t^{1/3})\\
                                                   &= s\bigg(\frac{1}{t^{1/3}}\bigg) = \frac{1}{t} \in C^{\infty}
    \end{align*}
    Onde na quarta igualdade usamos que $\pi_N \circ \pi_S^{-1}(x) : \R^* \to \R^* = 1/x$. A mesma conta serve para a outra composição $[s\circ \pi_S] \circ [s \circ \pi_N]^{-1}$.

    Vamos provar que $\mathcal{F} \neq \mathcal{G}$. Suponha que fossem iguais, então a composição $\pi_N \circ [s \circ \pi_N]^{-1} (t) = s^{-1} (t) = t^{1/3}$ 
    seria $C^{\infty}$ que sabemos que é falso. 

    Para provar que são diffeomorfas, considere: 
    \begin{align*}
        F : (S^1, \mathcal{F}) &\to (S^1, \mathcal{G})\\
            p \neq (0,1) &\mapsto (\pi_N^{-1}) \circ s^{-1} \circ \pi_N (p) \\
            p \neq (0,-1) &\mapsto (\pi_S^{-1}) \circ s^{-1} \circ \pi_S (p) \\
    \end{align*}
    Do jeito que está, $F$ pode não parecer bem definida. Seja $p \neq (0,1)$, $(0,-1)$. Queremos mostrar que: 
    \begin{equation}
        (\pi_N^{-1}) \circ s^{-1} \circ \pi_N (p) = (\pi_S^{-1}) \circ s^{-1} \circ \pi_S (p)
    \end{equation}
    Mas temos que todas as funções são homeomorfismos e, principalmente, $\pi_N \circ \pi_S^{-1} = 1/x$.
    Seja $\pi_N(p) = t$, então $t = [\pi_N \circ \pi_S^{-1}] \circ \pi_S (p) = 1/(\pi_S(p))$, ou seja $\pi_S(p) = 1/t$. Substituindo em (1)
    \begin{align*}
        (\pi_N^{-1}) \circ s^{-1}(t) &= (\pi_S^{-1}) \circ s^{-1}(1/t)\\
        s^{-1}(t) &= (\pi_N \circ \pi_S^{-1}) \circ s^{-1}(1/t)\\
        t^{1/3} &= \frac{1}{s^{-1}(1/t)} = t^{1/3}\\
    \end{align*}
    Onde na segunda igualdade aplicamos $\pi_N$ dos dois lados e na terceira usamos a composição usual. Como tudo pode ser feito de trás para frente,
    provamos que $F$ está bem definida.

    Agora basta provar que os seguintes mapas são diffeos $C^\infty$ em seus dominios (interseções das cartas):
    \begin{enumerate}
        \item $[s \circ \pi_N] \circ F \circ \pi_N^{-1}$
        \item $[s \circ \pi_N] \circ F \circ \pi_S^{-1}$
        \item $[s \circ \pi_S] \circ F \circ \pi_N^{-1}$
        \item $[s \circ \pi_S] \circ F \circ \pi_S^{-1}$
    \end{enumerate}
    E para isso é só expandi-los, farei (1) e (2) pois os outros dois são análogos.
    \begin{enumerate}
        \item $s \circ \pi_N \circ F \circ \pi_N^{-1} = s \circ \pi_N \circ (\pi_N^{-1}) \circ s^{-1} \circ \pi_N \circ \pi_N^{-1} = id$
        \item $s \circ \pi_N \circ F \circ \pi_S^{-1} = s \circ \pi_N \circ (\pi_N^{-1}) \circ s^{-1} \circ \pi_N \circ \pi_S^{-1} = 1/x$
    \end{enumerate}
\end{proof}


Para não perder nenhum detalhe, vou enunciar aqui a principal ferramenta desta lista.
\begin{theorem}
    \label{trm:partition_of_unity}
    Seja $M$ uma variedade diferenciável e $\{U_\alpha: \alpha \in A\}$ uma cobertura aberta de $M$. Então existe uma partição contável da unidade
    $\{\varphi_i : i \in \N\}$ subordinada a cobertura $\{U_\alpha\}$ com $\supp \varphi_i$ compacto para cada $i$. Se não for preciso suportes compactos,
    então existe uma partição da unidade $\{\varphi_\alpha\}$ subordinada à $\{U_\alpha\}$ ($\supp \varphi_\alpha \subset U_\alpha$) com no máximo
    contáveis $\varphi_\alpha$ não identicamente nulos.
\end{theorem}

\begin{problem}
\label{prob:l1:2}   
\end{problem}

\begin{proof}
    Pelo Teorema da Partição da Unidade \ref{trm:partition_of_unity}, dada uma cobertura $\{U_\alpha\}$, existe uma partição $\varphi_\alpha$ subordinada.
    Tome $V_\alpha = \varphi_\alpha^{-1}[(0,2)]$ abertos. Temos $\overline{V_\alpha} = \supp \varphi_\alpha \subset U_\alpha$ e, para todo $p \in M$, como $\sum_\alpha \varphi_\alpha(p) = 1$,
    existe $\alpha$ tal que $\varphi_\alpha(p) > 0$, logo $p \in V_\alpha$. Portanto $M \subset \{V_\alpha\}$ e temos um refinamento de $\{U_\alpha\}$.
\end{proof}

\begin{problem}
\label{prob:l1:3}   
\end{problem}

\begin{proof}
    Sejam $A$ e $B$ fechados disjuntos de $M$, então $\{A^c, B^c\}$ formam uma cobertura de $M$. Pelo Teorema da Partição da Unidade \ref{trm:partition_of_unity}, existem
    $\varphi_{A^c} \geq 0$ e $\varphi_{B^c} \geq 0$ em $C^\infty(M)$, com $\supp \varphi_{A^c} \subseteq A^c$ e  $\supp \varphi_{B^c} \subseteq B^c$. Como para todo $p \in M$, $\varphi_{A^c}(p) + \varphi_{B^c}(p) = 1$
    e $\varphi_{B^c} = 0$ em $B$, então temos
    \begin{align*}
        \varphi_{A^c}(A) = \{0\}\\
        \varphi_{A^c}(B) = \{1\}\\
    \end{align*}
    E achamos a segunda parte da questão, uma função contínua que vale $0$ em $A$ e $1$ em $B$.
    Tome então os abertos disjuntos $W_A = \varphi_{A^c}^{-1}[(-\infty,1/2)]$ e $W_B =\varphi_{A^c}^{-1}[(1/2,\infty)]$. Claramente $A \subset W_A$ e $B \subset W_B$.
\end{proof}

\begin{problem}
\label{prob:l1:4}   
\end{problem}
% Acredito não estar apto para a solução dessa questão, então, segue a solução que encontrei na internet para funções limitadas.

% A ideia da prova é construir uma sequência de funções contínuas que convirjam uniformemente em $M$ e se aproximam de $f$ em $A$.
% Como são funções contínuas convergindo uniformemente, o limite será contínuo em $M$. Para começar,
% precisaremos do seguinte lema.
% \begin{lemma}
%     \label{l1:lemma:approx}
%     Seja $A$ um fechado de $M$ e $f:A\to \R$ uma função limitada com $c = \sup_{a \in A} |f(a)|$. Então, existe uma função contínua $g : M \to \R$,
%     com $|g| \leq c/3$ em $M$ e $|f - g| \leq 2c/3$ em $A$.
% \end{lemma}
% \begin{proof}
%     Se $c = 0$, trivialmente, escolhemos $g = 0$. Se não, definimos os fechados disjuntos
%     \begin{align*}
%     E &= f^{-1}\{(-\infty, -c/3]\} \subseteq A\\
%     F &= f^{-1}\{(c/3,\infty)\} \subseteq A
%     \end{align*}
%     Pela questão anterior, como $E$ e $F$ são fechados disjuntos de $M$, existe $h:M \to [0,1]$ contínua, com $h(E) = \{0\}$ e $h(F) = \{1\}$.
%     Defina $g:M \to \R$ onde 
%     $$g(p) = h(p)\frac{2c}{3} - \frac{c}{3}$$
%     Então $g(E) = -c/3$, $g(F) = c/3$ e $g(p) \in [-c/3, c/3]$ para todo $p \in M$.

%     Dessa definição, é claro que $|g| \leq c/3$. Além disso $|f - g| \leq 2c/3$,
%     pois, se $p \in E$, 
%     $$c/3 = h(p) \leq f(p) \leq c \Rightarrow 0 \leq f - g \leq 2c/3$$
%     Se $p \in F$, semelhantemente
%     $$0 \leq g - f \leq 2c/3$$
%     E se $p$ não pertence a nenhum dos dois, então
%     $$f(p) \in [-c/3, c/3] \quad \land \quad g(p) \in [-c/3, c/3]$$
%     Logo $|f(p) - g(p)| \leq 2c/3$.
% \end{proof}

% \begin{prop}
% Dados $f$, $A$ e $M$ como na descrição do problema. Existe uma sequência de funções $\{g_n\}$ com 
% \begin{align*}
%     |g_n(x)| &\leq c\frac{2^n}{3^{n+1}} \quad \forall x \in M\\
%     |f - g_0 - \dots - g_n|(x) &:= |f - F_n(x)| \leq c\frac{2^{n+1}}{3^{n+1}} \quad \forall x \in A\\
% \end{align*}
% \end{prop}
% \begin{proof}
%     Aplique o lema \ref{l1:lemma:approx} em $f$ para obter $g_0$ satisfazendo as condições.  Depois aplicamos sucessivamente 
%     a $f - g_0$ obtendo $g_1$, depois $f - g_0 - g_1$, obtendo $g_2$ e etc. Formalizando com indução.
%     Suponha que conseguimos $(g_n)_{n=1}^{N}$, satisfazendo as condições, apliquemos o lema para
%     $$f - F_{N} = f - \sum_{n = 0}^{N} g_n$$
%     Note que, como os $f$ e os $g_n$ são limitados, podemos fazer isso após restrigir os $g_n$ em $A$. Ganhamos $g_{N+1}$ com
%     \begin{align*}
%         |g_{N+1}| &\leq \frac{\sup_{x \in A} |f - F_n(x)|}{3} \leq c \frac{2^{N+1}}{3^{N+2}}\\
%         |f - F_n - g_{N+1}| &\leq \frac{2\sup_{x \in A} |f - F_n(x)|}{3} \leq c \frac{2^{N+2}}{3^{N+2}}\\
%     \end{align*}
%     Confirmando a Hipótese de Indução.
% \end{proof}

% \begin{prop}
%     O somatório $\sum_{k=0}^{\infty} g_k$ é uma função contínua identica a $f$ em $A$.
% \end{prop}
% \begin{proof}
%     Basta notar que, como 
%     $$|g_n| \leq \frac{c}{3}\frac{2^n}{3^n}$$
%     A sequência converge geométricamente, portanto uniformemente e o somatório é contínuo.
%     Da mesma forma, vimos na proposição anterior que
%     $$|f(x) - \sum_{k=0}^{n}g_n(x)| \leq c\frac{2^{n+1}}{3^{n+1}} \to 0 \forall x \in A$$
%     Quando $n \to \infty$. Portanto a série é identica a $f$ em $A$.
% \end{proof}

Conversando na aula de segunda (25/08), perguntamos para o Prof. Heluani se deveríamos provar esse resultado (estender o acima para funções não limitadas). 
Ele nos disse que o interesse maior nesse problema era mostrar que esse resultado (Teorema da Extensão de Tietze) é válido para Variédades vistas como 
espaços topológicos. Isso é consequência de elas serem espaços normais (visto no problema anterior). Segue o enunciado do Teorema
\begin{theorem}
    (Extensão de Tietze) Seja $X$ um espaço topológico normal, $A \subseteq X$ um subconjunto fechado e $f : A \to \R$ uma função contínua. Então existe 
    uma função contínua $\tilde{f}: X \to \R$ tal que a $\tilde{f}\arrowvert_A = f$. 
\end{theorem}

Sobre a extensão ser $C^\infty$, devemos dar uma definição para o que isso significaria - uma função ser suave em um fechado. Aqui segui a ideia 
do Davi na monitoria, de que existe um abertinho maior em que ela está definida.
Depois devemos verificar se é possível estender funções assim para toda a variedade.
A resposta dessa afirmação é positiva, mas requer também um pouco mais de teoria do que vimos. A seguir temos uma tentativa.

\begin{lemma}
    \label{lemma:extensao_suave}
    Seja $M$ uma variedade, $A \subseteq M$ um conjunto fechado e $U \supseteq A$ um aberto onde está definida uma função 
    suave $f : U \to \R$. Existe uma função $\tilde{f} : M \to \R$ suave tal que as restrições $\tilde{f}_{A} = f_A$ são idênticas.
\end{lemma}
\begin{proof}
    Para cada ponto $p \in A$, escolha uma vizinhança e uma função suave $(V_p, \tilde{f}_p)$ tal que $\tilde{f}_p: V_p \to \R$ é idêntica a $f$ em $V_p \cap A$. 
    Isso é possível usando funções bump e aproveitando o fato que $M$ é localmente compacta - o que não foi provado. 
    Tomamos uma partição da unidade $\{\varphi_p : p \in A\} \cup \{\varphi_{A^c}\}$ subordinada a cobertura $\{V_p : p \in A\} \cup \{A^c\}$. 
    Para cada $p \in A$, o produto $\varphi_p \tilde{f}_p$ é $C^\infty$ em $V_p$ e tem uma extensão natural $0$ fora do 
    suporte de $\varphi_p$. Definimos então $\tilde{f} = \sum_{p\in A} \varphi_p \tilde{f}_p$.
\end{proof}

\end{document}
