\documentclass{article}

\usepackage{amsmath,amssymb,amsthm,bbm,mathtools,calc,verbatim,enumitem,tikz,url,mathrsfs,cite,fullpage,hyperref,bm,marvosym}
\usepackage{tikz-cd}
\usepackage{dsfont}
\usepackage{float}
\usepackage{subcaption}
%\usepackage{setspace}
\renewcommand{\baselinestretch}{1.1}
\addtolength{\footskip}{\baselineskip/2}

%\usepackage{showlabels}
\usepackage{comment}
\usepackage[english]{babel}

%No caso do livro o Tu, para decoplar seções de capitulos usamos:
% \usepackage{chngcntr}
% \counterwithout{section}{chapter}

\theoremstyle{definition}
\newtheorem{theorem}{Theorem}[section]
\newtheorem{lemma}[theorem]{Lemma}
\newtheorem{corollary}[theorem]{Corollary}
\newtheorem{prop}[theorem]{Proposition}
\newtheorem{observation}[theorem]{Observation}
\newtheorem{construction}[theorem]{Construction}

\newtheorem{definition}[theorem]{Definition}

\newtheorem{conjecture}[theorem]{Conjecture}
\newtheorem{question}[theorem]{Question}
\newtheorem{obs}[theorem]{Observation}
\newtheorem{claim}[theorem]{Claim}
\newtheorem{fact}[theorem]{Fact}
\newtheorem{problem}{Problem}[section]
\newtheorem{exercise}{Exercise}[section]
\newtheorem{remark}[theorem]{Remark}

% my custom problems
\newtheorem{innercustomexercise}{Exercise}
\newenvironment{customexercise}[1]
  {\renewcommand\theinnercustomexercise{#1}\innercustomexercise}
  {\endinnercustomexercise}

\newcounter{step}
\newenvironment{proofsteps}{
  \setcounter{step}{0}
  \begin{list}{\textbf{Step \Roman{step}.}}{\usecounter{step}
  \setlength{\leftmargin}{1.5em}
  \setlength{\itemsep}{0.3em}}
}{
  \end{list}
}

\newenvironment{clmproof}[1]{\begin{proof}[Proof of Claim~\ref{#1}]\let\qednow\qedsymbol\renewcommand{\qedsymbol}{}}{\; \qednow \end{proof}}

\newcommand\N{\mathbb{N}}
\newcommand\R{\mathbb{R}}
\newcommand\Z{\mathbb{Z}}
\newcommand\cA{\mathcal{A}}
\newcommand\cB{\mathcal{B}}
\newcommand\cN{\mathcal{N}}
\newcommand\cP{\mathcal{P}}
\newcommand\cQ{\mathcal{Q}}
\newcommand\cZ{\mathcal{Z}}
\newcommand\rN{\tilde{N}}
\newcommand\cT{\mathcal{T}}
\newcommand\cE{\mathcal{E}}
\def\Pr{\mathbb{P}}
\def\cS{\mathcal{S}}
\newcommand\Ex{\mathbb{E}}
\newcommand\id{\hbox{$1\mkern-6.5mu1$}}
\newcommand\lcm{\operatorname{lcm}}
\newcommand\eps{\varepsilon}
\newcommand{\floor}[1]{\lfloor #1 \rfloor}
\newcommand{\ceil}[1]{\lceil #1 \rceil}
\newcommand{\prob}{\begin{problem} \end{problem}}
\newcommand{\exer}{\begin{exercise} \end{exercise}}
\newcommand{\cexer}[1]{\begin{customprob}{#1}\end{customprob}}


\renewcommand{\leq}{\leqslant}
\renewcommand{\geq}{\geqslant}
\renewcommand{\le}{\leqslant}
\renewcommand{\ge}{\geqslant}
\renewcommand{\to}{\rightarrow}
\renewcommand{\Re}{\re}

\def\ds{\displaystyle}

\def\eps{\varepsilon}
\def\p{\partial}

\def\HH{\mathcal{H}}
\def\E{\mathbb{E}}
\def\C{\mathbb{C}}
\def\cM{\mathcal{M}}
\def\cF{\mathcal{F}}
\def\cI{\mathcal{I}}
\def\R{\mathbb{R}}
\def\bS{\mathbb{S}}
\def\bH{\mathbb{H}}
\def\Z{\mathbb{Z}}
\def\N{\mathbb{N}}
\def\PP{\mathbb{P}}
\def\1{\mathbbm{1}}
\def\l{}
\def\k{\kappa}
\def\w{\omega}
\def\s{\sigma}
\def\t{\theta}
\def\a{\alpha}
\def\g{\gamma}
\def\z{\zeta}
\def\zbar{\bar{z}}
\def\<{\langle}
\def\>{\rangle}
%\def\endproof{{\hfill $\square$} }
\def\Xt{\widetilde{X}}
\def\Pt{\widetilde{P}}

\def\cN{\mathcal{N}}
\def\cC{\mathcal{C}}
\def\cD{\mathcal{D}}
\def\cR{\mathcal{R}}
\def\cB{\mathcal{B}}
\def\cG{\mathcal{G}}
\def\EE{\mathbb{E}}
\def\FF{\mathbb{F}}
\def\T{\mathbb{T}}
\def\cA{\mathcal{A}}
\def\cQ{\mathcal{Q}}
\def\cC{\mathcal{C}}
\def\F{\mathbb{F}}
\def\tm{\tilde{\mu}}
\def\ts{\tilde{\sigma}}
\def\Q{\mathcal{Q}}
\def\vp{\varphi}

\DeclareMathOperator\supp{supp}

\hypersetup{
	colorlinks=true,
	linkcolor=blue,
	urlcolor=blue,
}

\pagestyle{plain}
\author{henrique}
\title{Listas de Variedades}

\begin{document}
\maketitle

\tableofcontents
\setcounter{section}{-1}

\section{Introdução e Notação}
Ao decorrer do curso, vou escrever minhas resoluções dos exercícios nesse arquivo. Tem alguns motivos para isso:
\begin{enumerate}
	\item Posso reutilizar resultados passados.
	\item Está tudo organizado se um futuro henrique quiser rever.
	\item Há uma certo senso de completude no final do curso.
\end{enumerate}
Por isso, peço desculpa ao monitor e ao professor se não gostarem desse formato, me avisem que eu posso separar os arquivos.
O código fonte pode ser encontrado em \url{https://github.com/hnrq104/variedades/tree/main/listas}.

Por enquanto encontrei os seguintes usos de notação pessoal no texto:
\begin{enumerate}
	\item Denoto $[n] = \{1, \dots, n \}$ o conjunto dos primeiros $n$ naturais.
\end{enumerate}

\section{Lista 1 (18/08/2025)}

Problemas feitos:
\begin{enumerate}
    \item Exercício \ref{prob:l1:1} : \checkmark
    \item Exercício \ref{prob:l1:2} : \checkmark
    \item Exercício \ref{prob:l1:3} : \checkmark
    \item Exercício \ref{prob:l1:4} : \Frowny
\end{enumerate}

\begin{problem}
\label{prob:l1:1}   
\end{problem}

\begin{proof}
    Defina $(S^1, \mathcal{F})$ a parametrização do círculo pelas projeções esfereográficas. Isto é,
    $$\mathcal{F} = \langle (S^1 - \{(0,1)\}, \pi_N), (S^1 - \{(0,-1)\}, \pi_S)\rangle$$
    Onde $\pi_N : S^1 - \{(0,1)\} \to \R$ e $\pi_S : S^1 - \{(0,-1)\} \to \R$ são as projeções do polo norte e sul respectivamente. Vimos em aula que, com
    essas coordenadas, $(S^1, \mathcal{F})$ é uma variedade $C^\infty$. Defina $\mathcal{G}$ elevando $\mathcal{F}$ ao cubo,
    $$\mathcal{G} = \langle(S^1 - \{(0,1)\}, (\pi_N)^3), (S^1 - \{(0,-1)\}, (\pi_S)^3)\rangle$$
    Afirmo que $\mathcal{G}$ é uma estrutura diferenciável de $S^1$. Isso segue do fato que $\pi_N^3$ e $\pi_S^3$ continuam sendo homeomorfismos 
    e a composição de cartas dão a mesma coisa que em $\mathcal{F}$. Para verificar isso, escreva $s(t) = t^3$, então, no intervalo de definição $\R^*$,
    \begin{align*}
        [(\pi_N)^{3}] \circ [(\pi_S)^{3}]^{-1} (t) &= (s\circ \pi_N) \circ (s \circ \pi_S)^{-1} (t)\\
                                                   &= s \circ \pi_N \circ \pi_S^{-1} \circ s^{-1} (t) \\
                                                   &= s \circ \pi_N \circ \pi_S^{-1} (t^{1/3})\\
                                                   &= s\bigg(\frac{1}{t^{1/3}}\bigg) = \frac{1}{t} \in C^{\infty}
    \end{align*}
    Onde na quarta igualdade usamos que $\pi_N \circ \pi_S^{-1}(x) : \R^* \to \R^* = 1/x$. A mesma conta serve para a outra composição $[s\circ \pi_S] \circ [s \circ \pi_N]^{-1}$.

    Vamos provar que $\mathcal{F} \neq \mathcal{G}$. Suponha que fossem iguais, então a composição $\pi_N \circ [s \circ \pi_N]^{-1} (t) = s^{-1} (t) = t^{1/3}$ 
    seria $C^{\infty}$ que sabemos que é falso. 

    Para provar que são diffeomorfas, considere: 
    \begin{align*}
        F : (S^1, \mathcal{F}) &\to (S^1, \mathcal{G})\\
            p \neq (0,1) &\mapsto (\pi_N^{-1}) \circ s^{-1} \circ \pi_N (p) \\
            p \neq (0,-1) &\mapsto (\pi_S^{-1}) \circ s^{-1} \circ \pi_S (p) \\
    \end{align*}
    Do jeito que está, $F$ pode não parecer bem definida. Seja $p \neq (0,1)$, $(0,-1)$. Queremos mostrar que: 
    \begin{equation}
        (\pi_N^{-1}) \circ s^{-1} \circ \pi_N (p) = (\pi_S^{-1}) \circ s^{-1} \circ \pi_S (p)
    \end{equation}
    Mas temos que todas as funções são homeomorfismos e, principalmente, $\pi_N \circ \pi_S^{-1} = 1/x$.
    Seja $\pi_N(p) = t$, então $t = [\pi_N \circ \pi_S^{-1}] \circ \pi_S (p) = 1/(\pi_S(p))$, ou seja $\pi_S(p) = 1/t$. Substituindo em (1)
    \begin{align*}
        (\pi_N^{-1}) \circ s^{-1}(t) &= (\pi_S^{-1}) \circ s^{-1}(1/t)\\
        s^{-1}(t) &= (\pi_N \circ \pi_S^{-1}) \circ s^{-1}(1/t)\\
        t^{1/3} &= \frac{1}{s^{-1}(1/t)} = t^{1/3}\\
    \end{align*}
    Onde na segunda igualdade aplicamos $\pi_N$ dos dois lados e na terceira usamos a composição usual. Como tudo pode ser feito de trás para frente,
    provamos que $F$ está bem definida.

    Agora basta provar que os seguintes mapas são diffeos $C^\infty$ em seus dominios (interseções das cartas):
    \begin{enumerate}
        \item $[s \circ \pi_N] \circ F \circ \pi_N^{-1}$
        \item $[s \circ \pi_N] \circ F \circ \pi_S^{-1}$
        \item $[s \circ \pi_S] \circ F \circ \pi_N^{-1}$
        \item $[s \circ \pi_S] \circ F \circ \pi_S^{-1}$
    \end{enumerate}
    E para isso é só expandi-los, farei (1) e (2) pois os outros dois são análogos.
    \begin{enumerate}
        \item $s \circ \pi_N \circ F \circ \pi_N^{-1} = s \circ \pi_N \circ (\pi_N^{-1}) \circ s^{-1} \circ \pi_N \circ \pi_N^{-1} = id$
        \item $s \circ \pi_N \circ F \circ \pi_S^{-1} = s \circ \pi_N \circ (\pi_N^{-1}) \circ s^{-1} \circ \pi_N \circ \pi_S^{-1} = 1/x$
    \end{enumerate}
\end{proof}


Para não perder nenhum detalhe, vou enunciar aqui a principal ferramenta desta lista.
\begin{theorem}
    \label{trm:partition_of_unity}
    Seja $M$ uma variedade diferenciável e $\{U_\alpha: \alpha \in A\}$ uma cobertura aberta de $M$. Então existe uma partição contável da unidade
    $\{\varphi_i : i \in \N\}$ subordinada a cobertura $\{U_\alpha\}$ com $\supp \varphi_i$ compacto para cada $i$. Se não for preciso suportes compactos,
    então existe uma partição da unidade $\{\varphi_\alpha\}$ subordinada à $\{U_\alpha\}$ ($\supp \varphi_\alpha \subset U_\alpha$) com no máximo
    contáveis $\varphi_\alpha$ não identicamente nulos.
\end{theorem}

\begin{problem}
\label{prob:l1:2}   
\end{problem}

\begin{proof}
    Pelo Teorema da Partição da Unidade \ref{trm:partition_of_unity}, dada uma cobertura $\{U_\alpha\}$, existe uma partição $\varphi_\alpha$ subordinada.
    Tome $V_\alpha = \varphi_\alpha^{-1}[(0,2)]$ abertos. Temos $\overline{V_\alpha} = \supp \varphi_\alpha \subset U_\alpha$ e, para todo $p \in M$, como $\sum_\alpha \varphi_\alpha(p) = 1$,
    existe $\alpha$ tal que $\varphi_\alpha(p) > 0$, logo $p \in V_\alpha$. Portanto $M \subset \{V_\alpha\}$ e temos um refinamento de $\{U_\alpha\}$.
\end{proof}

\begin{problem}
\label{prob:l1:3}   
\end{problem}

\begin{proof}
    Sejam $A$ e $B$ fechados disjuntos de $M$, então $\{A^c, B^c\}$ formam uma cobertura de $M$. Pelo Teorema da Partição da Unidade \ref{trm:partition_of_unity}, existem
    $\varphi_{A^c} \geq 0$ e $\varphi_{B^c} \geq 0$ em $C^\infty(M)$, com $\supp \varphi_{A^c} \subseteq A^c$ e  $\supp \varphi_{B^c} \subseteq B^c$. Como para todo $p \in M$, $\varphi_{A^c}(p) + \varphi_{B^c}(p) = 1$
    e $\varphi_{B^c} = 0$ em $B$, então temos
    \begin{align*}
        \varphi_{A^c}(A) = \{0\}\\
        \varphi_{A^c}(B) = \{1\}\\
    \end{align*}
    E achamos a segunda parte da questão, uma função contínua que vale $0$ em $A$ e $1$ em $B$.
    Tome então os abertos disjuntos $W_A = \varphi_{A^c}^{-1}[(-\infty,1/2)]$ e $W_B =\varphi_{A^c}^{-1}[(1/2,\infty)]$. Claramente $A \subset W_A$ e $B \subset W_B$.
\end{proof}

\begin{problem}
\label{prob:l1:4}   
\end{problem}
% Acredito não estar apto para a solução dessa questão, então, segue a solução que encontrei na internet para funções limitadas.

% A ideia da prova é construir uma sequência de funções contínuas que convirjam uniformemente em $M$ e se aproximam de $f$ em $A$.
% Como são funções contínuas convergindo uniformemente, o limite será contínuo em $M$. Para começar,
% precisaremos do seguinte lema.
% \begin{lemma}
%     \label{l1:lemma:approx}
%     Seja $A$ um fechado de $M$ e $f:A\to \R$ uma função limitada com $c = \sup_{a \in A} |f(a)|$. Então, existe uma função contínua $g : M \to \R$,
%     com $|g| \leq c/3$ em $M$ e $|f - g| \leq 2c/3$ em $A$.
% \end{lemma}
% \begin{proof}
%     Se $c = 0$, trivialmente, escolhemos $g = 0$. Se não, definimos os fechados disjuntos
%     \begin{align*}
%     E &= f^{-1}\{(-\infty, -c/3]\} \subseteq A\\
%     F &= f^{-1}\{(c/3,\infty)\} \subseteq A
%     \end{align*}
%     Pela questão anterior, como $E$ e $F$ são fechados disjuntos de $M$, existe $h:M \to [0,1]$ contínua, com $h(E) = \{0\}$ e $h(F) = \{1\}$.
%     Defina $g:M \to \R$ onde 
%     $$g(p) = h(p)\frac{2c}{3} - \frac{c}{3}$$
%     Então $g(E) = -c/3$, $g(F) = c/3$ e $g(p) \in [-c/3, c/3]$ para todo $p \in M$.

%     Dessa definição, é claro que $|g| \leq c/3$. Além disso $|f - g| \leq 2c/3$,
%     pois, se $p \in E$, 
%     $$c/3 = h(p) \leq f(p) \leq c \Rightarrow 0 \leq f - g \leq 2c/3$$
%     Se $p \in F$, semelhantemente
%     $$0 \leq g - f \leq 2c/3$$
%     E se $p$ não pertence a nenhum dos dois, então
%     $$f(p) \in [-c/3, c/3] \quad \land \quad g(p) \in [-c/3, c/3]$$
%     Logo $|f(p) - g(p)| \leq 2c/3$.
% \end{proof}

% \begin{prop}
% Dados $f$, $A$ e $M$ como na descrição do problema. Existe uma sequência de funções $\{g_n\}$ com 
% \begin{align*}
%     |g_n(x)| &\leq c\frac{2^n}{3^{n+1}} \quad \forall x \in M\\
%     |f - g_0 - \dots - g_n|(x) &:= |f - F_n(x)| \leq c\frac{2^{n+1}}{3^{n+1}} \quad \forall x \in A\\
% \end{align*}
% \end{prop}
% \begin{proof}
%     Aplique o lema \ref{l1:lemma:approx} em $f$ para obter $g_0$ satisfazendo as condições.  Depois aplicamos sucessivamente 
%     a $f - g_0$ obtendo $g_1$, depois $f - g_0 - g_1$, obtendo $g_2$ e etc. Formalizando com indução.
%     Suponha que conseguimos $(g_n)_{n=1}^{N}$, satisfazendo as condições, apliquemos o lema para
%     $$f - F_{N} = f - \sum_{n = 0}^{N} g_n$$
%     Note que, como os $f$ e os $g_n$ são limitados, podemos fazer isso após restrigir os $g_n$ em $A$. Ganhamos $g_{N+1}$ com
%     \begin{align*}
%         |g_{N+1}| &\leq \frac{\sup_{x \in A} |f - F_n(x)|}{3} \leq c \frac{2^{N+1}}{3^{N+2}}\\
%         |f - F_n - g_{N+1}| &\leq \frac{2\sup_{x \in A} |f - F_n(x)|}{3} \leq c \frac{2^{N+2}}{3^{N+2}}\\
%     \end{align*}
%     Confirmando a Hipótese de Indução.
% \end{proof}

% \begin{prop}
%     O somatório $\sum_{k=0}^{\infty} g_k$ é uma função contínua identica a $f$ em $A$.
% \end{prop}
% \begin{proof}
%     Basta notar que, como 
%     $$|g_n| \leq \frac{c}{3}\frac{2^n}{3^n}$$
%     A sequência converge geométricamente, portanto uniformemente e o somatório é contínuo.
%     Da mesma forma, vimos na proposição anterior que
%     $$|f(x) - \sum_{k=0}^{n}g_n(x)| \leq c\frac{2^{n+1}}{3^{n+1}} \to 0 \forall x \in A$$
%     Quando $n \to \infty$. Portanto a série é identica a $f$ em $A$.
% \end{proof}

Conversando na aula de segunda (25/08), perguntamos para o Prof. Heluani se deveríamos provar esse resultado (estender o acima para funções não limitadas). 
Ele nos disse que o interesse maior nesse problema era mostrar que esse resultado (Teorema da Extensão de Tietze) é válido para Variédades vistas como 
espaços topológicos. Isso é consequência de elas serem espaços normais (visto no problema anterior). Segue o enunciado do Teorema
\begin{theorem}
    (Extensão de Tietze) Seja $X$ um espaço topológico normal, $A \subseteq X$ um subconjunto fechado e $f : A \to \R$ uma função contínua. Então existe 
    uma função contínua $\tilde{f}: X \to \R$ tal que a $\tilde{f}\arrowvert_A = f$. 
\end{theorem}

Sobre a extensão ser $C^\infty$, devemos dar uma definição para o que isso significaria - uma função ser suave em um fechado. Aqui segui a ideia 
do Davi na monitoria, de que existe um abertinho maior em que ela está definida.
Depois devemos verificar se é possível estender funções assim para toda a variedade.
A resposta dessa afirmação é positiva, mas requer também um pouco mais de teoria do que vimos. A seguir temos uma tentativa.

\begin{lemma}
    \label{lemma:extensao_suave}
    Seja $M$ uma variedade, $A \subseteq M$ um conjunto fechado e $U \supseteq A$ um aberto onde está definida uma função 
    suave $f : U \to \R$. Existe uma função $\tilde{f} : M \to \R$ suave tal que as restrições $\tilde{f}_{A} = f_A$ são idênticas.
\end{lemma}
\begin{proof}
    Para cada ponto $p \in A$, escolha uma vizinhança e uma função suave $(V_p, \tilde{f}_p)$ tal que $\tilde{f}_p: V_p \to \R$ é idêntica a $f$ em $V_p \cap A$. 
    Isso é possível usando funções bump e aproveitando o fato que $M$ é localmente compacta - o que não foi provado. 
    Tomamos uma partição da unidade $\{\varphi_p : p \in A\} \cup \{\varphi_{A^c}\}$ subordinada a cobertura $\{V_p : p \in A\} \cup \{A^c\}$. 
    Para cada $p \in A$, o produto $\varphi_p \tilde{f}_p$ é $C^\infty$ em $V_p$ e tem uma extensão natural $0$ fora do 
    suporte de $\varphi_p$. Definimos então $\tilde{f} = \sum_{p\in A} \varphi_p \tilde{f}_p$.
\end{proof}
\section{Lista 2 (02/09/2025)}

Problemas feitos:
\begin{enumerate}
    \item Exercício \ref{prob:l2:1} : \Frowny
    \item Exercício \ref{prob:l2:2} : \checkmark
    \item Exercício \ref{prob:l2:3} : \Frowny
    \item Exercício \ref{prob:l2:4} : \Frowny
    \item Exercício \ref{prob:l2:5} : \Frowny
\end{enumerate}

\begin{problem}
    \label{prob:l2:1}    
\end{problem}
\begin{proof}
    Considere o atlas de $S^1 \subset \C$ gerado por $(U, \theta_1)$ e $(V, \theta_2)$
    onde  $U = S^1 - \{1\}$ e $\theta_1$ é definida tomando o ramo apropriado do logaritmo de forma que
    \begin{align*}
        \theta_1 :U &\to (0,2\pi)\\
            z &\mapsto \frac{\log(z)}{i} 
    \end{align*}
    Semelhantemente, $V = S^1 - \{-1\}$ e escolhemos um outro ramo de $\log$ a fim que
    \begin{align*}
        \theta_2 : V &\to (-\pi,\pi)\\
            z &\mapsto \frac{\log(z)}{i} 
    \end{align*}
    Sabemos que esses ramos $\log$ são diffeos em seus domínios e a composição
    $\theta_1 \circ \theta_2^{-1} : (-\pi,0) \cup (0,\pi)$
    dada por 
    $$
        \theta_1 \circ \theta_2^{-1} (x) =
        \begin{cases}
        x & \text{se } 0 < x < \pi\\
        x + \pi & \text{se } 0 < x < \pi\\
        \end{cases}
    $$
    Essa função é $C^\infty$, pois os abertos da definição são disjuntos. Da mesma forma $\theta_2 \circ \theta_1^{-1} \in C^\infty$.

    Para mostrar que $e^{ix} : \R \to S^1$ é $C^\infty$, basta ver que a composições com os mapas é $C^\infty$. Para 
    todo $x \in \R$, existe $n \in \Z$ tal que
    $$x \in (2\pi n, 2\pi(n+1)) \quad \lor \quad x \in (2\pi n - \pi, 2\pi(n+1) - \pi)$$
    No primeiro caso, claramente $\theta_1(e^{ix}) = x - 2\pi n$ que é $C^\infty$. No segundo caso, 
    $\theta_2(e^{ix}) = x - 2\pi n - \pi \in \C^\infty$. Assim mostramos que para qualquer
    ponto de $\R$, existe um aberto $U$ tal que a composição $\theta_i \circ e^{ix}\arrowvert_U$ é $C^\infty$,
    como consequência, $e^{ix}$ é $C^\infty$.
\end{proof}


\begin{problem}
    \label{prob:l2:2} 
\end{problem}
\begin{proof}
    Tome o mesmo atlas que na questão anterior, note que se $(U, \theta)$ é uma carta e $V \subset U$, 
    então $(V, \theta\arrowvert_V)$ é óbviamente uma carta do atlas também.
    Para mostrar que $z^2 \in C^\infty$ vamos verificar então que para todo $z \in S^1$, existe um uma carta $(A, \theta)$ ao redor de $z$
    e uma carta $(B, \phi)$ ao redor de $z^2$ tal que a composição $\phi \circ z^2 \circ \theta^{-1}$ é $C^\infty$.
    
    Separamos $4$ cartas de $S^1$ e qual coordenadas usaremos na imagem de cada vizinhança,
    \begin{enumerate}
        \item $(A_1 = \{z : \text{Re}(z) > 0\}$, $\theta_2\arrowvert_{A_1}$) e  $(B_1 = V, \theta_2)$
        \item $(A_2 = \{z : \text{Im}(z) > 0\}$, $\theta_1\arrowvert_{A_2}$) e  $(B_2 = U, \theta_1)$ 
        \item $(A_3 = \{z : \text{Re}(z) < 0\}$, $\theta_1\arrowvert_{A_3}$) e  $(B_3 = U, \theta_1)$ 
        \item $(A_4 = \{z : \text{Im}(z) < 0\}$, $\theta_2\arrowvert_{A_4}$) e  $(B_4 = V, \theta_2)$  
    \end{enumerate}
    Claramente os $A_i$ cobrem $S^1$ e como para cada $i$, $z^2(A_i) = B_i$, estamos cobrindo a imagem de  $z^2$.
    Substituindo os índices, falta verificar que em cada item que a composição $\theta_k \circ z^2 \circ (\theta_k\arrowvert_{A_i})^{-1}$ é $C^\infty$.
    Calculando-as, temos
    \begin{enumerate}
        \item $\theta_2 \circ z^2 \circ (\theta_2\arrowvert_{A_1})^{-1} : (-\pi/2, \pi/2) \to (-\pi, \pi)$ é tal que $z \mapsto 2z$
        \item $\theta_1 \circ z^2 \circ (\theta_1\arrowvert_{A_2})^{-1} : (0, \pi) \to (0, 2\pi)$ é tal que $z \mapsto 2z$
        \item $\theta_1 \circ z^2 \circ (\theta_1\arrowvert_{A_3})^{-1} : (\pi/2, 3\pi/2) \to (-\pi, \pi)$ é tal que $z \mapsto 2z - 2\pi$
        \item $\theta_2 \circ z^2 \circ (\theta_2\arrowvert_{A_4})^{-1} : (-\pi, 0) \to (0, 2\pi)$ é tal que $z \mapsto 2z + 2\pi$
    \end{enumerate}
    Como todas são $C^\infty$, $z^2$ é $C^\infty$.

\end{proof}

Para os próximos dois problemas, vamos enunciar a ferramenta principal e sua versão em variedades - presumo que ainda será bastante 
utilizada no curso.
\begin{theorem}
    (Teorema da Função Implícita) Seja $U \subset \R^{c -d}\times \R^d$ aberto e $f: U \to \R^d \in C^\infty$. Denote o sistema canônico de coordenadas em
    $\R^{c -d}\times \R^d$ por $(r_1, \dots, r_{c-d}, s_1, \dots, s_d)$. Suponha que no ponto $(r_0,s_0) \in U$
    $$f(r_0, s_0) = 0$$
    e que a matriz
    $$\bigg\{\frac{\partial f_i}{\partial s_j}\bigg\}_{i,j = 1,\dots,d}$$
    é não singular. Então existe uma vizinhança aberta $V$ de $r_0$ em $R^{c-d}$ e uma vizinhança aberta $W$ de $s_0$ em $\R^d$
    tal que $V \times W \subset U$, e existe um mapa $C^\infty$ $g:V\to W$ tal que para cada par $(p,q) \in V\times W$
    $$f(p,q) = 0 \iff q = g(p)$$ 
\end{theorem}

\begin{theorem}
    Assuma que $\psi: M^c \to N^d$ é $C^\infty$, que $n$ é um ponto de $N$, tal que $P = \psi^{-1}(n)$ é não vazio, e 
    que $d\psi: M_m \to N_{\psi(m)}$ é sobrejetiva para todo $m \in P$. Então $P$ tem uma estrutura diferenciável única
    tal que $(P,i)$ é subvariedade de $M$, onde $i$ é o mapa da inclusão. Além do mais, $i : P \to M$ é uma imersão, e a dimensão
    de $P$ é $c - d$.
\end{theorem}

\begin{problem}
    \label{prob:l2:3} 
\end{problem}
\begin{proof}
\end{proof}


\begin{problem}
    \label{prob:l2:4} 
\end{problem}
Eu sei resolver esse problema de duas formas, a mais natural para mim caiu na minha prova em Análise no Rn e 
eu falhei feio, então me lembro bem da solução. Esse método, no entanto, usa uma versão um pouco diferente do 
Teorema da função Implícita e fatos sobre a derivada de Frechet que fogem ao escopo deste curso, vou escrevê-lo porque 
acredito ser uma solução válida também. A outra forma de fazer tem uma delineação no Warner em 1.40 - essa é a que presumo
que o professor esteja pedindo. De qualquer forma, escreverei ambas. 
\begin{proof}
    
\end{proof}

\begin{problem}
    \label{prob:l2:5}
\end{problem}
\begin{proof}
    
\end{proof}
\clearpage
\section{Lista 3 (15/10/25)}

Como não consegui responder quase nenhum problema completamente, não vou colocar a lista de problemas resolvidos.

\begin{theorem}
    \label{trm:quotient_manifolds} (3.58 Warner) Seja $H$ um subgrupo fechado do grupo de Lie $G$. Seja $\pi : G \to G/H$ a projeção
    $\pi(g) = gH$. Então $G/H$ tem uma única estrutura diferenciável tal que $\pi$ é $C^\infty$.
\end{theorem}

\begin{theorem}
    \label{trm:isotropy_manifolds} (3.62 Warner) Seja $G \times M \to M$ uma ação transitiva do grupo de Lie $G$ na variedade $M$. Seja $m \in M$
    e $H$ o grupo de isotropia de $m$. O mapa 
    $$\beta: G/H \to M \quad \beta(gH) = gm_0$$
    é um difeomorfismo. 
\end{theorem}


\begin{exercise}
    \label{prob:l3:1}
\end{exercise}
\begin{proof}
    Vou mostrar algo um pouco mais geral, seja $V$ um $R$ espaço vetorial $n$ dimensional, $G_k(V)$ o conjunto 
    de subespaços $k$ dimensionais de $V$, então 
    $$G_k(V) \cong \frac{O(n)}{O(k) \times O(n-k)}.$$
    Em particular, $G_2(R^4)$ = $O(4)/(O(2)\times O(2))$ de onde seguirá que ela é compacta de dimensão $4$.
    
    Em $V$, fixe uma base $E_1, \dots E_n$. As matrizes $O(n)$ agem em $V$ por multiplicação e mandam $k$-subespaços
    vetoriais em $k$-subespaços vetoriais (não vou provar isso). Dessa forma, $O(n)$ age naturalmente em $G_k(V)$.
    É outro fato da vida (fácil de ver em $R^n$ com Gram-Schmidt) que para $V_0$ e $V_1$ dois $k$-subespaços de $V$
    existem transformações ortogonais que mandam $V_0$ em $V_1$, portanto a ação $O(n)$ é transitiva. 

    Seja $P = \langle E_1, \dots E_k \rangle$ e $H < O(n)$ o grupo de isotropia de $P$. Se $A \in H$, então como $A$ fixa $P$,
    $A$ deve mandar os vetores $E_1, \dots, E_k$ em $\langle E_1, \dots E_k \rangle$ e, sendo invertível, os vetores $E_{k+1} \dots E_{n}$
    em $\langle E_{k+1}, \dots E_n \rangle$. Portanto já determinamos que A deve ser da forma 
    $$A = \begin{pmatrix}
        F && 0\\
        0 && H \\
    \end{pmatrix}$$
    onde $F$ é uma matrix $k\times k$ e $H$ é $(n-k)\times(n-k)$. Como $A^tA = I$,
    segue que 
    $$I = A^tA = \begin{pmatrix}
        F^t && 0\\
        0 && H^t \\
    \end{pmatrix}
    \begin{pmatrix}
        F && 0\\
        0 && H \\
    \end{pmatrix}
    =
    \begin{pmatrix}
        F^tF && 0\\
        0 && H^tH\\
    \end{pmatrix}
    $$
    e portanto $F^tF = I$ e $H^tH = I$, ou seja $F \in O(k)$ e $H \in O(n-k)$. Interessantemente, qualquer escolha 
    de $F$ e $G$ em $O(k) \times O(n-k)$ fixa $P$ (satisfaz as condições), então podemos identificar $H$ justamente como $O(k)\times O(n-k)$.
    Note que $H$ é um grupo fechado, é a interseção de pré-imagems de fechados por funções contínuas; dado $B \in O(n)$,
    escrevemos 
    $$B = \begin{pmatrix}
        B_{11} && B_{12}\\
        B_{21} && B_{22}\\
    \end{pmatrix}$$
    e $H$ se torna 
    \begin{align*}
        H &= \{B \in O(n) : B_{11}^tB_{11} = I\} \cap \{B \in O(n) : B_{12} = \vec{0}\}\\ 
        & \cap \,\{B \in O(n) : B_{22}^tB_{22} = I\} \cap \{B \in O(n) : B_{21} = \vec{0}\}
    \end{align*}
    Portanto, podemos dar uma estrutura diferenciável para $G_k(V)$ correspondente a $O(n)/(O(k)\times O(n-k))$.
    Segue por [\ref{trm:quotient_manifolds}] que $O(n)/H$ é a imagem 
    por uma função suave $\pi : O(n) \to O(n)/H$ de $O(n)$, portanto $O(n)/H = G_k(V)$ é imagem de compacto logo compacto.
    
    
    Para calcular a dimensão de $G_2(R^4)$, lembramos que $\dim(O(4)) = 4\cdot3/2 = 6$ e $\dim(O(2)) = 1$, 
    portanto $\dim(O(4)/(O(2) \times O(2))) = 6 - (1 + 1) = 4$. 

    Sobre ser diffeomorfa com $S^2 \times S^2$, eu não sei provar. Pelo jeito a respota é negativa.
\end{proof}

\begin{exercise}
    \label{prob:l3:2}
\end{exercise}
\begin{proof}
    Seguirei a solução do David bonitinha de fazer na mão as cartas da variedade.
    
    Antes de mais nada, vamos mostrar que $Q$ é compacto. Como $f$, é contínua, $f^{-1}({0})$ é um fechado de $\C^4 - \vec{0}$, como o mapa do quociente
    $\pi : \C^4 - \{\vec{0}\} \to \C P^3$ é fechado, segue que $\pi(f^{-1}(\{0\}))$ é um fechado de $\C P^3$, como $\C P^3$ é compacto, segue que $Q$ é compacto.

    Agora vamos dar uma estrutura diferenciável para $Q$ - ele já é N2 pela topologia induzida.
    Dada uma carta de $\C P^3$, digamos $U_1 = \{x_1 \neq 0\}$ com coordenadas $\varphi_1([1:x_2:x_3:x_4]) = (x_2,x_3,x_4) \in \C^3$,
    é evidente que $U_1 \cap Q$ são os pontos satisfazendo $g_1([1:x_2:x_3:x_4)] = x_4 - x_2x_3 = 0$. Olhando $h(x_2,x_3,x_4) = g_1\circ\varphi^{-1}$ como uma função 
    em $\C^3$, ela tem rank constante (basta olhar para $x_4$) e portanto, $M_1 = h^{-1}(\{0\})$ é um embedding em $\C^3$.

    Seja $\psi$ uma carta qualquer de $M_1$, daremos a $Q \cap U_1$ as cartas definidas pela composição de cartas $\psi \circ \varphi_1$, 
    \[
    Q \cap U_1 \xrightarrow{\ \varphi_1\ } M_1 \subset \C^3 \xrightarrow{\ \psi\ } W \subset \C^2
    \]
    Como $M_1$ é um embedding, $\psi^{-1}(W) = B \cap M_1$, onde $B$ é um aberto de $\C^3$, definimos a carta em $Q$
    por $(Q \cap \varphi_1^{-1}(B), \psi \circ \varphi_1)$.
    
    A construção anterior de cartas para $Q$ pode ser repetida para $U_2 = \{x_2 \neq 0\}, U_3 = \{x_3 \neq 0\}$ e $U_4 = \{x_4 \neq 0\}$.
    A menos de troca de variáveis (e sinal) as funções $h$ são identicas para cada um desses abertos de $\C P^3$. Completamos nossas 
    coordenadas de $Q$ com as $\psi \circ \varphi_i$ definidas em cada $Q \cap U_i$.

    Falta mostrar que troca de coordenadas é $C^\infty$, seja $F_{ij} = \psi_j \circ \varphi_j \circ \varphi_i^{-1} \circ \psi_i^{-1}$
   \[
        \begin{tikzcd}[column sep=huge, row sep=huge]
        Q \arrow[r, "\varphi_i"] \arrow[dr, "\varphi_j"'] 
        & \C^3 \arrow[d, dashed, "\varphi_j \circ \varphi_i^{-1}"]  \supset M_i \arrow[r, "\psi_i"] 
            & W_i \arrow[d, dashed, "F_{ij}"] \\
        & \C^3 \supset M_j \arrow[r, "\psi_j"'] & W_j
        \end{tikzcd}
    \]
    Como composição de funções $C^\infty$, (e o fato de $\psi_j$ serem embeddings para a continuidade), segue que $F_{ij}$ é $C^\infty$.
    E portanto $Q$ é uma variedade diferenciável compacta de dimensão complexa 2 (e dimensão real 4).

    Sobre ser diffeomorfa com $S^2 \times S^2$, acho que o David falou que era, não tenho ideia de como provar.



\end{proof}

\begin{exercise}
    \label{prob:l3:3}
    Essa questão é copiar palavra por palavra a solução do Warner. Não acho que vale a pena escrever. Se fizer, farei por último.
\end{exercise}

\begin{exercise}
    \label{prob:l3:4}
\end{exercise}
\begin{proof}
    A primeira coisa a notar é que, como subconjuntos de $GL(2,\R)$ e $GL(3,\R)$, $SL(2,\R)$ não é compacto,
    pois 
    $$\begin{pmatrix}
        1/n & 0 \\
        0 & n \\
    \end{pmatrix} \in SL(2,\R)$$
    para todo $n$, portanto, não é limitado. Já $SO(3,\R)$ é um fechado do compacto $O(3,\R)$ então é compacto. 
    E já temos que $SL(2,\R)$ não pode ser diffeomorfa a $SO(3,\R)$.

    Seja $F: M \to N$ de rank constante e $S \subset M$ embedding dado por $S = F^{-1}(0)$ - pelo Teorema do Rank Constante.
    Sabemos que para $p \in S$, $T_p S = \ker(dF_p)$, usaremos isso para calcular $\mathfrak{so}(3,\R)$ e $\mathfrak{sl}(2,\R)$.

    Já que $SO(3,\R)$ é um subgrupo aberto de $O(3,\R)$, $T_ISO(3,\R) = T_IO(3,\R)$. Então, para calcular $\mathfrak{so}(3,\R)$,  $\mathfrak{o}(3,\R)$ 
    que sabemos ser o kernel da transformação
    $$A \in GL(3,\R) \xrightarrow{\ \Phi\ } A^tA \in M(3,\R)$$
    Já calculamos $d\Phi_I$ e vimos que
    $$d\Phi_I(B) = B + B^t$$
    portanto, $\mathfrak{so}(3,\R) = \ker(d\Phi_I) = \{B \in M(3,\R): B = - B^t\}$, 
    ou seja $\mathfrak{so}(3,\R)$ é a algebra de Lie das matrizes $3\times 3$ antissimétricas com o colchete 
    sendo o comutador (por que é subálgebra de $Lie(GL(3,\R))$).

    Lembrando que $SL(2,\R) = \det^{-1}(\{1\}) \subset GL(2,\R)$ e que $d(\det)_I(B) = \text{tr}(B)$, calculamos, pela mesma observação 
    anterior que $\mathfrak{sl(2,\R)} = \{B \in M(2,\R): \text{tr}(B) = 0\}$ com o colchete do comutador novamente.
    
    Vamos dar bases para as duas álgebras de Lie e mostrar que elas não são isomorfas pela observação do Roger.

    Uma base para $\mathfrak{so}(3,\R)$ pode ser 
    $$A = \begin{pmatrix}
        0 & 1 & 0\\
        -1 & 0 & 0\\
        0 & 0 & 0\\
    \end{pmatrix}
    \quad
    B = \begin{pmatrix}
        0 & 0 & -1\\
        0 & 0 & 0\\
        1 & 0 & 0\\
    \end{pmatrix}
    \quad
    C = \begin{pmatrix}
        0 & 0 & 0\\
        0 & 0 & 1\\
        0 & -1 & 0\\
    \end{pmatrix}$$
    E depois de um cálculo meio chato dos colchetes (lembrando que $C^t = -C$ e $(FG) = (F^tG^t)^t$ nessas matrizes)
    temos 
    $$[A,B] = C \quad [B,C] = A \quad [A,C] = -B$$
    Note que o colchete de dois vetores L.I leva em um terceiro L.I aos dois.
    
    Uma base para $\mathfrak{sl}(2,\R)$ é dada por
    $$E = \begin{pmatrix}
        1 & 0 \\
        0 & -1
    \end{pmatrix}
    \quad
    F = \begin{pmatrix}
        0 & 1 \\
        0 & 0 
    \end{pmatrix}
    \quad
    G = \begin{pmatrix}
        0 & 0 \\
        1 & 0 
    \end{pmatrix}$$
    
    Calculando os colchetes, achamos
    $$[E,F] = 2F \quad [F,G] = E \quad [E,G] = -2G $$
    Mas então, o colchete em $\mathfrak{sl(2,\R)}$ leva dois vetores em L.I (digamos $E,F$) em um vetor L.D com os dois $2F$.
    Portanto $\mathfrak{so}(3,\R) \not \cong \mathfrak{sl}(2,\R).$
\end{proof}
    
\begin{exercise}
    \label{prob:l3:5}
\end{exercise}
\begin{proof}
    Para calcular $\mathfrak{sl}(2,\C)$ e $\mathfrak{so}(3,\C)$, exatamente as mesmas contas anteriores servem, mas,
    curiosamente, sobre os complexos, as algebras de Lie são isomorfas. Usando os mesmos vetores que na questão anterior
    e definindo $T:\mathfrak{so}(3,\C) \to \mathfrak{sl}(2,\C)$ por 
    \begin{align*}
        A \quad &\mapsto \quad \frac{i}{2}(F + G)\\
        B \quad &\mapsto \quad \frac{1}{2}(G - F)\\
        C \quad &\mapsto \quad \frac{i}{2}(E)
    \end{align*}
    Vemos que $T$ manda linearmente uma base em outra, logo é invertível. Falta verificar que é homomorfismo de álgebras de Lie,
    para isso temos que verificar que preserva o colchete, segue as computações necessárias. 
    $$[TA, TB] = [(F + G)i/2, (G - F)/2] = ([F,G] + [G,-F])i/4 = [F,G]i/2 = Ei/2 = TC$$
    $$[TB, TC] = [(G - F)/2, Ei/2] = ([G,E] - [F,E])i/4 = (2G + 2F)i/4 = (G + F)i/2 = TA$$
    $$[TC, TA] = [Ei/2, (F + G)i/2] = -([E,F] + [E,G])/4 = -(2F - 2G)/4 = (G - F)/2 = TB$$
    segue que $T$ é homomorfismo. Sobre se $SO(3,\C)$ e $SL(2,\C)$ serem diffeomorfas ou não, não 
    sei fazer. 
\end{proof}

\begin{exercise}
    \label{prob:l3:6}
\end{exercise}
\begin{proof}
    \textbf{(1.a)} É fácil ver que $D$ é suave de dimensão 2, pois podemos escolher 
    os campos suaves em $SO(3)$ associados a $e_1$ e $e_2$ em $\mathfrak{so}(3)$.
    Mais precisamente, os campos suaves $d(L_g)e_1$ e $d(L_g)e_2$ definidos em toda variedade por 
    definição geram $D$ pontualmente, logo $D$ é suave.

    \textbf{(1.b)} Como os campos são associados a translação, vale que 
    $$d(L_g)([X,Y]) = [d(L_g)X, d(L_g)Y]$$
    portanto basta checar na identidade. Mas vimos que $[e_1, e_2] = e_3 \not \in \langle e_1, e_2 \rangle$,
    portanto a distribuição não é involutiva.

    \textbf{(1.c)} Pelo teorema de Frobënius, D é integrável se e somente se for involutiva, como não é 
    involutiva, não é integrável.

    \textbf{(2)} Como $SO(3,\R) \subset GL(3,\R)$ é subgrupo de Lie, para encontrar uma curva integral de $SO(3,\R)$, basta aplicar a exponenciação
    em $GL(3,\R)$. A curva integral dada por $X = e_1$ começando no ponto $g$ será $g \cdot \exp(t e_1)$
    
    Identificando $\C$ com o grupo de matrizes 
    $$a + ib = \begin{pmatrix}
        0 & 0 & 0 \\
        0 & a & -b\\
        0 & b & a
    \end{pmatrix}$$
    é fácil verificar que $e_1$ é mandado em $i$ e $\exp(e_1t)$ terá exatamente o mesmo papel 
    que $\exp(it)$. Isso é só uma forma fácil de ver que, expandindo a definição de $\exp(e_1 t)$
    teremos algo do tipo
    $$\exp(e_1 t) = \begin{pmatrix}
        0 & 0 & 0 \\
        0 & \sin(t) & -\cos(t) \\
        0 & \cos(t) & \sin(t)
    \end{pmatrix}$$
    (que pode ser explicitamente verificado expandindo a conta em cada coordenada e usando as expansões de Taylor).
    Segue que as curvas integrais são fechadas com período $2\pi$. Já sabemos que ela deveriam ser completas pois $SO(3)$ é compacto
    e portanto fechadas (por serem imagens de $\R$). Mas verifica-se justamente que as curvas são rotações sobre o primeiro eixo em $SO(3)$. 
\end{proof}




\end{document}
