\clearpage
\section{Lista 2 (02/09/2025)}

Problemas feitos:
\begin{enumerate}
    \item Exercício \ref{prob:l2:1} : \checkmark
    \item Exercício \ref{prob:l2:2} : \checkmark
    \item Exercício \ref{prob:l2:3} : \checkmark
    \item Exercício \ref{prob:l2:4} : \checkmark
    \item Exercício \ref{prob:l2:5} : \checkmark
\end{enumerate}

\begin{problem}
    \label{prob:l2:1}    
\end{problem}
\begin{proof}
    Considere o atlas de $S^1 \subset \C$ gerado por $(U, \theta_1)$ e $(V, \theta_2)$
    onde  $U = S^1 - \{1\}$ e $\theta_1$ é definida tomando o ramo apropriado do logaritmo de forma que
    \begin{align*}
        \theta_1 :U &\to (0,2\pi)\\
            z &\mapsto \frac{\log(z)}{i} 
    \end{align*}
    Semelhantemente, $V = S^1 - \{-1\}$ e escolhemos um outro ramo de $\log$ a fim que
    \begin{align*}
        \theta_2 : V &\to (-\pi,\pi)\\
            z &\mapsto \frac{\log(z)}{i} 
    \end{align*}
    Sabemos que esses ramos $\log$ são diffeos em seus domínios e a composição
    $\theta_1 \circ \theta_2^{-1} : (-\pi,0) \cup (0,\pi)$
    dada por 
    $$
        \theta_1 \circ \theta_2^{-1} (x) =
        \begin{cases}
        x & \text{se } 0 < x < \pi\\
        x + \pi & \text{se } 0 < x < \pi\\
        \end{cases}
    $$
    Essa função é $C^\infty$, pois os abertos da definição são disjuntos. Da mesma forma $\theta_2 \circ \theta_1^{-1} \in C^\infty$.

    Para mostrar que $e^{ix} : \R \to S^1$ é $C^\infty$, basta ver que a composições com os mapas é $C^\infty$. Para 
    todo $x \in \R$, existe $n \in \Z$ tal que
    $$x \in (2\pi n, 2\pi(n+1)) \quad \lor \quad x \in (2\pi n - \pi, 2\pi(n+1) - \pi)$$
    No primeiro caso, claramente $\theta_1(e^{ix}) = x - 2\pi n$ que é $C^\infty$. No segundo caso, 
    $\theta_2(e^{ix}) = x - 2\pi n - \pi \in \C^\infty$. Assim mostramos que para qualquer
    ponto de $\R$, existe um aberto $U$ tal que a composição $\theta_i \circ e^{ix}\arrowvert_U$ é $C^\infty$,
    como consequência, $e^{ix}$ é $C^\infty$.
\end{proof}


\begin{problem}
    \label{prob:l2:2} 
\end{problem}
\begin{proof}
    Tome o mesmo atlas que na questão anterior, note que se $(U, \theta)$ é uma carta e $V \subset U$, 
    então $(V, \theta\arrowvert_V)$ é óbviamente uma carta do atlas também.
    Para mostrar que $z^2 \in C^\infty$ vamos verificar então que para todo $z \in S^1$, existe um uma carta $(A, \theta)$ ao redor de $z$
    e uma carta $(B, \phi)$ ao redor de $z^2$ tal que a composição $\phi \circ z^2 \circ \theta^{-1}$ é $C^\infty$.
    
    Separamos $4$ cartas de $S^1$ e qual coordenadas usaremos na imagem de cada vizinhança,
    \begin{enumerate}
        \item $(A_1 = \{z : \text{Re}(z) > 0\}$, $\theta_2\arrowvert_{A_1}$) e  $(B_1 = V, \theta_2)$
        \item $(A_2 = \{z : \text{Im}(z) > 0\}$, $\theta_1\arrowvert_{A_2}$) e  $(B_2 = U, \theta_1)$ 
        \item $(A_3 = \{z : \text{Re}(z) < 0\}$, $\theta_1\arrowvert_{A_3}$) e  $(B_3 = U, \theta_1)$ 
        \item $(A_4 = \{z : \text{Im}(z) < 0\}$, $\theta_2\arrowvert_{A_4}$) e  $(B_4 = V, \theta_2)$  
    \end{enumerate}
    Claramente os $A_i$ cobrem $S^1$ e como para cada $i$, $z^2(A_i) = B_i$, estamos cobrindo a imagem de  $z^2$.
    Substituindo os índices, falta verificar que em cada item que a composição $\theta_k \circ z^2 \circ (\theta_k\arrowvert_{A_i})^{-1}$ é $C^\infty$.
    Calculando-as, temos
    \begin{enumerate}
        \item $\theta_2 \circ z^2 \circ (\theta_2\arrowvert_{A_1})^{-1} : (-\pi/2, \pi/2) \to (-\pi, \pi)$ é tal que $z \mapsto 2z$
        \item $\theta_1 \circ z^2 \circ (\theta_1\arrowvert_{A_2})^{-1} : (0, \pi) \to (0, 2\pi)$ é tal que $z \mapsto 2z$
        \item $\theta_1 \circ z^2 \circ (\theta_1\arrowvert_{A_3})^{-1} : (\pi/2, 3\pi/2) \to (-\pi, \pi)$ é tal que $z \mapsto 2z - 2\pi$
        \item $\theta_2 \circ z^2 \circ (\theta_2\arrowvert_{A_4})^{-1} : (-\pi, 0) \to (0, 2\pi)$ é tal que $z \mapsto 2z + 2\pi$
    \end{enumerate}
    Como todas são $C^\infty$, $z^2$ é $C^\infty$.

\end{proof}

Para os próximos dois problemas, vamos enunciar a ferramenta principal e sua versão em variedades - presumo que ainda será bastante 
utilizada no curso.
\begin{theorem}
    (Teorema da Função Implícita) Seja $U \subset \R^{c -d}\times \R^d$ aberto e $f: U \to \R^d \in C^\infty$. Denote o sistema canônico de coordenadas em
    $\R^{c -d}\times \R^d$ por $(r_1, \dots, r_{c-d}, s_1, \dots, s_d)$. Suponha que no ponto $(r_0,s_0) \in U$
    $$f(r_0, s_0) = 0$$
    e que a matriz
    $$\bigg\{\frac{\partial f_i}{\partial s_j}\bigg\}_{i,j = 1,\dots,d}$$
    é não singular. Então existe uma vizinhança aberta $V$ de $r_0$ em $R^{c-d}$ e uma vizinhança aberta $W$ de $s_0$ em $\R^d$
    tal que $V \times W \subset U$, e existe um mapa $C^\infty$ $g:V\to W$ tal que para cada par $(p,q) \in V\times W$
    $$f(p,q) = 0 \iff q = g(p)$$ 
\end{theorem}

\begin{theorem}
    \label{trm:implicit_function}
    Assuma que $\psi: M^c \to N^d$ é $C^\infty$, que $n$ é um ponto de $N$, tal que $P = \psi^{-1}(n)$ é não vazio, e 
    que $d\psi: M_m \to N_{\psi(m)}$ é sobrejetiva para todo $m \in P$. Então $P$ tem uma estrutura diferenciável única
    tal que $(P,i)$ é subvariedade de $M$, onde $i$ é o mapa da inclusão. Além do mais, $i : P \to M$ é uma imersão, e a dimensão
    de $P$ é $c - d$.
\end{theorem}

\begin{problem}
    \label{prob:l2:3} 
\end{problem}
\begin{proof}
    Vamos provar que $\text{SL}(n,\R)$ é uma subvariedade de $\R^{n\times n}$ invocando o teorema anterior com a função $\det : \R^{n\times n} \to \R$.
    Para isso, basta mostrar que $1$ é valor regular de $\det$, ou seja, se $A \in \text{SL}(n,\R)$, então $d(\det)\arrowvert_A$ é sobrejetiva em $\R$.
    Como o contradomínio é $\R$, basta mostrar que em uma coordenada $x_{ij}$ vale que 
    $$\frac{\partial \det}{\partial x_{ij}}\bigg\arrowvert_A \neq 0$$
    Mas isso segue da decomposição de Cramer do determinante. Seja $X_{i,j}^c$ a matriz dos cofatores de uma matriz $X$ em $i,j$. Por Cramer, para qualquer $k \in [n]$,
    \begin{equation}
        \det(X) = \sum_{i = 1}^n (-1)^{k+i} x_{k,i} \det(X_{k,i}^c)
    \end{equation}
    Naturalmente,
    $$\frac{\partial \det}{\partial x_{k,j}}\bigg\arrowvert_X = (-1)^{k+j} \det(X_{k,i}^c)$$
    Como $\det(A) = 1$, aplicando a regra em $k = 1$, segue que existe alguma matriz de cofatores $A_{1,i}^c$ com determinante
    não nulo. Portanto,
    $$\frac{\partial \det}{\partial x_{1,i}}\bigg\arrowvert_A = (-1)^{1+i} \det(A_{1,i}^c) \neq 0$$
    e o diferencial é sobrejetivo. Por \ref{trm:implicit_function}, $\text{SL}(n,\R)$ é uma imersão em $\R^{n\times n}$ de dimensão 
    $n^2 - 1$.
\end{proof}


\begin{problem}
    \label{prob:l2:4} 
\end{problem}
% Eu sei resolver esse problema de duas formas, a mais natural para mim caiu na minha prova em Análise no Rn e 
% eu falhei feio, então me lembro bem da solução. Esse método, no entanto, usa uma versão um pouco diferente do 
% Teorema da função Implícita e fatos sobre a derivada de Frechet que fogem ao escopo deste curso, vou escrevê-lo porque 
% acredito ser uma solução válida também. A outra forma de fazer tem uma delineação no Warner em 1.40 - essa é a que presumo
% que o professor esteja pedindo. De qualquer forma, escreverei ambas. 
\textbf{Primeira} prova usual rápida.
\begin{proof}
    Considere $F: \text{GL}(d,\R) \to \R^{d(d+1)/2}$ tal que
    $$F(A) = A\cdot A^{T}$$
    Note que como $(A \cdot A^T)^T = A\cdot A^T$, $F(A)$ é simétrica e o contradomínio faz sentido.
    Vamos provar que $I$ (como matriz simétrica) é valor regular da função $F$ para que possamos aplicar \ref{trm:implicit_function}.
    A diferencial de $F$ em $A$ em relação a um vetor $H$ pode ser obtida facilmente considerando a diferença (que é linear) de $F(A + H) - F(A)$.
    $$DF_A(H) = F(A + H) - F(A) = H\cdot A^T + A \cdot H^T$$
    Para mostrar que essa diferencial é sobrejetiva, tome $B$ tal que $B = B^T$, então escolhendo $H = BA/2$, temos que 
    $$DF_A(BA/2) = \frac{(BA)A^T}{2} + \frac{A(BA)^T}{2} = \frac{B}{2} + \frac{B^T}{2} = B$$
    Onde usamos que $AA^T = I$ na segunda igualdade. Portanto, $DF_A$ é sobrejetiva para cada $A \in F^{-1}(I)$, logo, 
    pelo Teorema da Função Implícita, $F^{-1}(I)$ é uma subvariedade de $\text{GL}(d,\R)$ de dimensão $d(d-1)/2$.
\end{proof}

\textbf{Segunda} prova, delineada pelo Warner em 1.40(b).
\begin{proof}
    Novamente, seja $F$ como na prova anterior. Usaremos a mesma estratégia, queremos mostrar 
    que $dF_A$ é sobrejetiva para cada $A \in O(d)$. Dado $A \in O(d)$, defina o diffeo $r_A : \text{GL}(n,\R) \to \text{GL}(n,\R)$ 
    que para $M \in \text{GL}(n,\R)$,
    $$r_A(M) = M\cdot A$$
    Quando se espressa $d(r_A)$ em coordenadas é claro que - como função linear, $d(r_A) = A$, logo é $C^\infty$. Da mesma forma, a inversa 
    $r_{A^{-1}}$ é $C^\infty$. Aqui está a ideia principal do Warner, note que 
    $$F = F \circ r_A$$
    pois, para todo $X$,
    $$F(X) = XX^T = (XA)(XA)^T = F \circ r_A (X)$$
    Portanto podemos diferenciar a função da direita ao invés de somente $F$, usando a regra da cadeia, encontramos que
    $$dF_A = d(F \circ r_{A^{-1}})_A = dF_{I} \circ d(r_{A^{-1}})_A = dF_{I} \circ A^{-1}$$
    Como $A^{-1}$ é invertível, para saber se $dF_A$ é sobrejetiva, basta verificar se $dF_I$ é sobrejetiva.
    Mas isso seguirá imediatamente do fato que para $i \leq j$ e $l \leq k$,
    $$\frac{\partial F_{i,j}}{\partial x_{l,k}}\bigg\arrowvert_I = \delta^{i,j}_{l,k}$$
    Isso é, a derivada de $F$ em $I$ em relação a coordenada $(l,k)$ é justamente a matriz simétrica com $1$ somente na 
    coordenada $(l,k)$ e $0$ nas outras entradas.
    Para a prova, note que
    \begin{equation}
        F(X)_{i,j} = \sum_{m = 1}^{d} x_{i,m}  x_{j,m}
    \end{equation}
    Logo, se $l,k = i,j$,
    $$\frac{\partial F_{i,j}}{\partial x_{i,j}}\bigg\arrowvert_{X = I} = x_{j,j}|_I = 1$$
    Se $l \neq i$,
    $$\frac{\partial F_{i,j}}{\partial x_{l,k}} = 0$$
    Pois $x_{l,k}$ sequer aparece na soma (3). Se $l = i$, mas $k \neq j$,
    então
    $$\frac{\partial F_{i,j}}{\partial x_{i,k}}\bigg\arrowvert_{X = I} = x_{j,k}|_I = 0$$
    E provamos a a identidade que queriámos. Portanto, $dF_I$ é sobrejetiva, por consequência, $dF_A$ é sobrejetiva 
    para todo $A \in O(d)$, logo $O(d)$ é naturalmente subvariedade de $\text{GL}(n,\R)$.


\end{proof}


\begin{problem}
    \label{prob:l2:5}
\end{problem}
Esse problema é capcioso, eu infelizmente já havia visto a solução antes da lista estar disponível.
\begin{proof}
    Como $M$ é compacto, a imagem $f(M)$ é compacta, logo existe $m \in M$ tal que a primeira coordenada $f_1(m)$ atinge o valor máximo.
    Dado $(x_i)_{i=1}^d$ um sistema de coordenadas na vizinhança de $m$,  como $f_1(m)$ é máximo, para cada $i$
    $$\frac{\partial f_1}{\partial x_i} = 0$$
    Portanto a matriz jacobiana 
    $$\bigg\{ \frac{\partial f_i}{\partial x_j}\bigg\}_{i,j \in [d]}$$
    é composta de $0$'s na primeira linha. $df$ não pode ser sobrejetiva.  
\end{proof}